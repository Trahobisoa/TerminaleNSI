\documentclass[french]{article}

\usepackage[T1]{fontenc}
\usepackage[utf8]{inputenc}
\usepackage{lmodern}
\usepackage[a4paper]{geometry}
\geometry{hmargin=2cm,vmargin=2cm}
\usepackage{babel}

\usepackage{soul}												% pour surligner les textes





\usepackage{graphicx}										% pour insérer des images

\usepackage[table]{xcolor}								  % pour colorier les celulles d'un tableau

\usepackage{tikz}										% pour faire des graphes


\usepackage{amssymb}        								% permet les énumérations avec symboles   

\usepackage{tabularx}											% permet de faire des tableaux

\usepackage{wrapfig}											% permet de mettre des images encadrées de texte




\usepackage{hyperref}     										% permet les liens hypertextes


\hypersetup{
	colorlinks=true,
	linkcolor=blue,    % Couleur des liens internes
	citecolor=green,   % Couleur des liens vers des citations
	urlcolor=red       % Couleur des liens vers des URL
}

%-----------------------------------------------------
\usepackage{listings}
\lstset{																													% permet d'écrire du code en console
	basicstyle=\ttfamily,
	frame=single, % Encadrement simple
	breaklines=true % Permet le retour à la ligne automatique
}
%-----------------------------------------------------

\lstset{
	language=Python,
	basicstyle=\ttfamily,
	numbers=left,
	numberstyle=\tiny\color{gray},
	keywordstyle=\color{blue},
	commentstyle=\color{red},
	stringstyle=\color{purple},	
	backgroundcolor=\color{lightgray},														% permet d'écrire du code en Python
	breaklines=true,
	breakatwhitespace=true,
	showspaces=false,
	showstringspaces=false,
	tabsize=4
}


\usepackage[]{tcolorbox}

\usepackage[]{enumitem}

\usepackage[]{lipsum}

\usepackage[]{multicol}		% permet d'afficher sur plusieurs colonnes

\usepackage{parskip}	


% gestion des entetes et pied de page
%-----------------------------------------------------
\usepackage{fancyhdr}
\usepackage{lastpage}
\pagestyle{fancy}
\fancyhf{}
\fancyhead[L]{\small \textit{Terminale - Spécialité NSI}}
\fancyhead[R]{\small \textit{IFS Singapour}}
\renewcommand\footrulewidth{1pt}
\fancyfoot[L]{\small \textit{T. Rahobisoa}}
\fancyfoot[C]{\thepage/\pageref{LastPage}}
\fancyfoot[R]{\small \textit{SD05 - Les arbres }}
%-----------------------------------------------------


\setlength{\parindent}{0pt}   % permet d'enlever l'indentation en début de paragraphe

%-----------------------------------------------------
\usepackage{ntheorem}
\theoremstyle{plain}
\theorembodyfont{\normalfont}													% permet d'écrire des "Exercices" et "A faire" numérotés
\theoremseparator{~:}
\newtheorem{exo}{Exercice}%[section]
\newtheorem{afaire}{A faire}%[section]

%-----------------------------------------------------


%-----------------------------------------------------
\usepackage{titlesec}
\titleformat{\section}
{\normalfont\Large\bfseries}{\thesection.}{1em}{}           % permet de mettre un point après la numérotation des sections
%-----------------------------------------------------

\usepackage{tikz}\usetikzlibrary{shapes.misc}

\newcommand\titlebar{%
	\tikz[baseline,trim left=3.1cm,trim right=3cm] {
		\fill [cyan!25] (2.5cm,-1ex) rectangle (\textwidth+3.1cm,2.5ex);
		\node [
		fill=cyan!60!white,
		anchor= base east,
		rounded rectangle,															% permet d'écrire les sections dans des cadres colorés
		minimum height=3.5ex] at (2.9cm,0) {
			\textbf{\thesection}
		};
	}%
}
\titleformat{\section}{\Large\bfseries}{\titlebar}{0.3cm}{}
\renewcommand*{\thesection}{\arabic{section}}


%-----------------------------------------------------



\usepackage[ruled,vlined,french]{algorithm2e}   % permet d'écrire des algorithmes en pseudo-code


\newcommand\gauchedroite[2]{\noindent{}#1\hfill#2}

\usepackage{mdframed}

\date{2023-2024}
\author{Tantely Rahobisoa}



\begin{document}
	
	
	\begin{tcolorbox}[colback=blue!5!white,colframe=black!75!black]
	%	\gauchedroite{Langage et programmation}{IFS Singapour, NSI 2023-2024}\par
	\begin{center}
		\LARGE{\textbf{Chapitre 10 - Les arbres}}
	\end{center}
\end{tcolorbox}

\vspace{0.5cm}

\subsection*{Objectifs :}

\begin{itemize}[label=$\triangleright$]
	\item Identifier des situations nécessitant une structure de données arborescente.
	\item Connaître le vocabulaire des arbres : noeud, racine, feuille, hauteur, taille, ..., \textit{etc.}
	\item Évaluer quelques mesures des arbres binaires (taille, encadrement de la hauteur, etc.).
	
\end{itemize}


\section{Introduction}

Il n'existe pas seulement une façon \textbf{linéaire} de structurer les données comme avec les listes, les piles ou les files. Nous pouvons également structurer des données de façon \textbf{hiérarchique}. 

Par exemple, lorsqu'on utilise un \textbf{arbre de classification des animaux} :

	\begin{figure}[h]  % Utilisation de [h] pour indiquer que l'image doit être placée ici (si possible)
	\centering
	\includegraphics[width=0.8\textwidth]{animaux.jpg}
	\end{figure}


Ou encore, lorsqu'on crée le très classique \textbf{arbre généalogique} :

	\begin{figure}[h]  % Utilisation de [h] pour indiquer que l'image doit être placée ici (si possible)
	\centering
	\includegraphics[width=0.67\textwidth]{family.jpg}
	\end{figure}

Parfois, on utilise même ce mode de représentation graphique pour l'apprentissage d'une langue.

Un \textbf{arbre syntaxique} représente l'analyse d'une phrase à partir de règles de grammaire. Par exemple : 

	\begin{figure}[h]  % Utilisation de [h] pour indiquer que l'image doit être placée ici (si possible)
	\centering
	\includegraphics[width=0.7\textwidth]{syntaxique.jpg}
	\end{figure}
	
	Il existe aussi des \textbf{arbres lexicographiques}. Appelé aussi \textbf{dictionnaire}, il représente un ensemble de mots. Les préfixes communs à plusieurs mots apparaissent une seule fois dans l'arbre.
	
		\begin{figure}[h]  % Utilisation de [h] pour indiquer que l'image doit être placée ici (si possible)
		\centering
		\includegraphics[width=0.52\textwidth]{lexicographique.png}
		\end{figure}


Dans les systèmes d'exploitation (\textit{Windows}, \textit{MacOS}, \textit{Linux}), le disque dur est représenté et organisé suivant une \textbf{arborescence} comme ci-dessous :


	\begin{figure}[h]  % Utilisation de [h] pour indiquer que l'image doit être placée ici (si possible)
	\centering
	\includegraphics[width=0.75\textwidth]{arborescence.png}
	\end{figure}

\newpage

On peut également représenter les \textbf{expressions arithmétiques} par des arbres étiquetés par des opérateurs, des constantes et des variables.

La structure de l'arbre rend compte de la priorité des opérateurs et rend inutile tout parenthésage.

	\begin{figure}[h]  % Utilisation de [h] pour indiquer que l'image doit être placée ici (si possible)
	\centering
	\includegraphics[width=0.25\textwidth]{arithmetique.png}
\end{figure}

Quelle est l'expression arithmétique représentée par cet arbre ?

\begin{tcolorbox}[colback=blue!5!white,colframe=blue!75!black,title=Réponse :]
	
\vspace*{1cm}
	
\end{tcolorbox}

\vspace*{0.3cm}

En utilisant le même principe, représenter l'expression suivante : $$\left(\frac{y}{2} - t\right) \times (75 + z)$$


\begin{tcolorbox}[colback=blue!5!white,colframe=blue!75!black,title=Réponse :]
	
	\vspace*{9cm}
	
\end{tcolorbox}

\vspace*{0.3cm}

Les arbres sont très utilisés en informatique, d'une part parce que les informations sont souvent \textbf{hiérarchisées} et peuvent se représenter naturellement sous une forme arborescente, et d'autre part, parce que les structures de données arborescentes permettent de stocker des données \textbf{volumineuses} de façon que leur accès soit \textbf{efficace}.

\newpage

\section{Notions générales sur les arbres}

On peut considérer un arbre comme une généralisation d'une liste car les listes peuvent être représentées par des arbres. Plutôt que de chercher à définir ce qu'est un arbre, nous observerons quelques schémas :

	\begin{figure}[h]  % Utilisation de [h] pour indiquer que l'image doit être placée ici (si possible)
	\centering
	\includegraphics[width=0.3\textwidth]{foret.png}
	\caption{Représentation graphique d'une forêt}
	\includegraphics[width=0.3\textwidth]{arbre.png}
	\caption{Représentation graphique d'un arbre}
	\end{figure}

Attention, la représentation graphique ci-dessous n'est pas celle d'un arbre car il existe un chemin d'un sommet vers lui-même (appelé un cycle).

	\begin{figure}[h]  % Utilisation de [h] pour indiquer que l'image doit être placée ici (si possible)
	\centering
	\includegraphics[width=0.27\textwidth]{cycle.png}
	\caption{Représentation graphique d'un cycle}
	\end{figure}

Lorsqu'un sommet se distingue des autres, on le nomme \textbf{racine} de l'arbre et celui-ci devient alors une \textbf{arborescence} (par la suite on utilisera le mot arbre pour une arborescence).

Les 3 arbres ci-dessous représentent la même structure, cependant pour deux d'entre eux, un sommet peut être désigné comme racine.

	\begin{figure}[h]  % Utilisation de [h] pour indiquer que l'image doit être placée ici (si possible)
	\centering
	\includegraphics[width=0.88\textwidth]{racines.png}
	\caption{Racines d'un arbre}
	\end{figure}
	
	Lorsqu'on dessine un arbre, on a l'habitude de le représenter avec la tête en bas, c'est-à-dire que \textbf{la racine est tout en haut}, et les nœuds fils sont représentés en dessous du nœud père.
	
	\section{Définitions et vocabulaire}
	
	
		\subsection{Etiquette}
		
		\begin{tcolorbox}[colback=red!5!white,colframe=red!75!black,title=A retenir !]
			
Un arbre dont tous les nœuds sont nommés est dit étiqueté. L'\textbf{étiquette} (ou nom du sommet) représente la \textbf{valeur} du nœud ou bien l'\textbf{information associée au nœud}.
			
		\end{tcolorbox}
		
		
			\begin{figure}[h]  % Utilisation de [h] pour indiquer que l'image doit être placée ici (si possible)
			\centering
			\includegraphics[width=0.45\textwidth]{etiquette.png}
		\end{figure}
		
		Ci-dessus un arbre étiqueté avec les entiers entre 2 et 8. 
		
			\subsection{Racine, noeud, branche, feuille}
			
		\begin{tcolorbox}[colback=red!5!white,colframe=red!75!black,title=A retenir !]
	
\begin{itemize}[label=$\triangleright$]
	\item Un arbre est un ensemble organisé de \textbf{nœuds} dans lequel chaque nœud a un père, sauf un nœud que l'on appelle la \textbf{racine}. 
	\item Si le nœud  n'a pas de fils, on dit que c'est une \textbf{feuille}.
	\item Les nœuds sont reliés par des \textbf{branches}.
\end{itemize}

	\end{tcolorbox}
	
				\begin{figure}[h]  % Utilisation de [h] pour indiquer que l'image doit être placée ici (si possible)
		\centering
		\includegraphics[width=0.53\textwidth]{vocabulaire.png}
	\end{figure}
	
	\newpage
	
	\subsection{Hauteur d'un noeud}
	
			\begin{tcolorbox}[colback=red!5!white,colframe=red!75!black,title=A retenir !]
		
La \textbf{hauteur} (ou \textbf{profondeur} ou \textbf{niveau} ) d'un nœud X est égale au nombre d'arêtes qu'il faut parcourir à partir de la racine pour aller jusqu'au nœud X.
		
	\end{tcolorbox}
	
Par convention, la hauteur (ou profondeur) de la racine est égale à 0. Mais attention, la définition de la hauteur d'un nœud varie en fonction des auteurs. Pour certains, la racine a une hauteur de 1.


\begin{figure}[h]  % Utilisation de [h] pour indiquer que l'image doit être placée ici (si possible)
	\centering
	\includegraphics[width=0.6\textwidth]{hauteur.png}
\end{figure}

Dans l'exemple ci-dessus la hauteur du noeud 9 est de 3 et celle du noeud 7 est de 2.

	\begin{tcolorbox}[colback=red!5!white,colframe=red!75!black,title=A retenir !]
	
La hauteur (ou profondeur) d'un arbre est égale à la profondeur du nœud le plus profond.
	
\end{tcolorbox}

Dans notre exemple, le nœud le plus profond est de profondeur 3, donc l'arbre est de profondeur 3.

\vspace*{0.5cm}

\textbf{Question 1 :} Déterminer les profondeurs des arbres de l'introduction :

\begin{tcolorbox}[colback=blue!5!white,colframe=blue!75!black,title=Réponses :]
	
\begin{itemize}[label=$\triangleright$]
	\item Arbre de classification :
	\item Arbre généalogique :
	\item Arbre syntaxique :
	\item Arbre lexicographique :
	\item Arborescence de disque dur :
\end{itemize}
	
\end{tcolorbox}

\subsection{Taille d'un arbre}

\begin{tcolorbox}[colback=red!5!white,colframe=red!75!black,title=A retenir !]
	
La \textbf{taille} d'un arbre est égale au nombre de nœuds de l 'arbre.
	
\end{tcolorbox}

Dans notre exemple, l'arbre contient 10 nœuds, sa taille est donc de 10.

\newpage

\textbf{Question 2 :} Déterminer les tailles des arbres de l'introduction :

\begin{tcolorbox}[colback=blue!5!white,colframe=blue!75!black,title=Réponses :]
	
	\begin{itemize}[label=$\triangleright$]
		\item Arbre de classification :
		\item Arbre généalogique :
		\item Arbre syntaxique :
		\item Arbre lexicographique :
		\item Arborescence de disque dur :
	\end{itemize}
	
\end{tcolorbox}


\subsection{Remarques}

Le vocabulaire de lien entre nœuds de niveaux différents et reliés entre eux est emprunté à la \textbf{généalogie}.

Dans l'exemple précédent :

\begin{itemize}[label=$\triangleright$]
	\item 8 est le parent de 9 et de 10
	\item 6 est un enfant de 4
	\item 6 et 7 sont des noeuds frères
	\item 5 est un ancêtre de 9
	\item 10 est un descendant de 5
\end{itemize}

Un arbre dont tous les nœuds n'ont qu'un seul fils est en fait une \textbf{liste}.

\begin{figure}[h]  % Utilisation de [h] pour indiquer que l'image doit être placée ici (si possible)
	\centering
	\includegraphics[width=0.33\textwidth]{liste.png}
\end{figure}

\end{document}
